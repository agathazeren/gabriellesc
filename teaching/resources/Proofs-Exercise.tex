\documentclass{article}

\usepackage{fullpage}
\usepackage{textcomp}
\usepackage{amsmath}

\usepackage{fancyhdr}
\pagestyle{fancy}
\renewcommand{\headrulewidth}{0pt}
\cfoot{\sc Page \thepage\ of \pageref{end}}

\begin{document}

{\large \noindent{}University of Toronto at Scarborough\\
\textbf{CSC A67/MAT A67 - Discrete Mathematics, Fall 2015}}

\section*{\huge Exercise \#7: Proofs}

{\large Due: November 20, 2015 at 11:59 p.m.\\
This exercise is worth 3\% of your final grade.}\\[1em]
\textbf{Warning:} Your electronic submission on MarkUs affirms that this exercise is your own work and no
one else's, and is in accordance with the University of Toronto Code of Behaviour on Academic Matters,
the Code of Student Conduct, and the guidelines for avoiding plagiarism in CSC A67/MAT A67.\\[1ex]
This exercise is due by 11:59 p.m. November 20. Late exercises will not be accepted.\\[1ex]
\renewcommand{\labelenumi}{\arabic{enumi}.}
\renewcommand{\labelenumii}{(\alph{enumii})}
\begin{enumerate}
\item The\marginpar{[5]} \textit{greatest common divisor} of two positive integers $a$ and $b$ is the largest positive integer that divides both $a$ and $b$ (written $\gcd(a,b)$). For example, $\gcd(4,6)=2$ and $\gcd(5,6)=1$.
	\begin{enumerate}
	\item Prove that $\gcd(a,b) = \gcd(a,b-a).$
	\item Let $r=b\bmod a$. Using part \textbf{(a)}, prove that $\gcd(a,b) = \gcd(a,r)$.
	\end{enumerate}
\item Prove\marginpar{[4]} that $\sqrt[3]{5}$ is irrational.
\item Prove\marginpar{[4]} the following statement by contraposition:\\[1ex]
Let $x$ be an integer. If $x^2+x+1$ is even, then $x$ is odd.
\item Prove\marginpar{[4]} the following statement by contradiction:\\[1ex]
Let $x$ and $y$ be integers. If $3x+5y=153$, then at least one of $x$ and $y$ is odd.
\item An\marginpar{[4]} integer is called ``sane" if $3\,|\,(n^2+2n)$. (That is, if $(n^2+2n)\bmod 3=0$.)
	\begin{enumerate}
	\item Prove or disprove that all odd integers are sane.
	\item Prove or disprove that, if $3\,|\,n$, then $n$ is sane.
	\end{enumerate}
\end{enumerate}
\hrulefill\\
\noindent[Total: 21 marks]\label{end}

\end{document}