\documentclass{article}

\usepackage{fullpage}
\usepackage{amsmath,amssymb}

\setlength{\parindent}{0pt}

\begin{document}

\section*{The Birthday Problem}

\textsc{Q: What is the probability that, in a group of $n$ people, at least 2 have the same birthday?}\\[1em]
Let us represent the days of the year by the integers 1, 2, \ldots, 365. Then we choose our sample space $\textsl{S}$ to be \{all possible combinations of $n$ birthdays\}. That is, we include all possible combinations of $n$ days, with repetition (up to $n$ repetitions of the same day, where all $n$ birthdays fall on the same day).\\[1ex]
For example, if $n=3$, we include:
\begin{itemize}
\item all the single days of the year (eg. (1, 1, 1), (2, 2, 2), (3, 3, 3), \ldots), in the case that all 3 birthdays fall on the same day,
\item all combinations of 2 different days of the year (eg. (1, 1, 2), (1, 1, 3), (1, 1, 4), \ldots), in the case that 2 of the birthdays fall on the same day, and
\item all combinations of 3 different days of the year (eg. (1, 2, 3), (1, 2, 4), (1, 2, 5), \ldots), in the case that all 3 birthdays fall on different days
\end{itemize}
Suppose that all birthdays are equally likely. Then, by the classical definition of probability,
\begin{equation*}
\boxed{
P(\{\text{at least 2 people share a birthday}\})=\dfrac{|\{\text{at least 2\ldots}\}|}{|\textsl{S}|}
}
\end{equation*}
We know from counting principles that
\begin{align*}
|\textsl{S}|& =\text{\# of ways to select the first birthday $\times$ \# of ways to select the second birthday}\\
& \quad\times\ldots\times\text{\# of ways to select the $n$th birthday}\\
& =365\times 365\times\ldots\times 365\\
& =365^n
\end{align*}
and that
\begin{align*}
|\{\text{at least 2\ldots}\}|& =\text{\# of arrangements of $n$ birthdays where 2 people share a birthday}\\
& \quad+\text{\# of arrangements of $n$ birthdays where 3 people share a birthday}\\
& \quad+\ldots+\text{\# of arrangements of $n$ birthdays where $n$ people share a birthday}
\end{align*}
So
\begin{equation*}
\boxed{
P(\{\text{at least 2 people share a birthday}\})=\dfrac{|\{\text{2 people share a birthday}\}|+\ldots+|\{\text{$n$ people share a birthday}\}|}{365^n}
}
\end{equation*}
Then we consider that
\begin{align*}
\text{\# of arrangements of $n$ birthdays}& \\
\text{where $r$ people share a birthday}& =\text{\# of ways to select the first unshared birthday}\\
& \quad\times\text{\# of ways to select the second unshared birthday}\\
& \quad\times\ldots\times\text{\# of ways to select the $(n-r)$th unshared birthday}\\
& \quad\times \text{\# of ways to select the shared birthday}\\
& \quad\times \text{\# of ways to arrange $n-r$ different birthdays and $r$ same birthdays}\\
& =365\times 364\times\ldots\times(365-(n-r-1))\times(365-(n-r))\times \dfrac{n!}{r!1!\ldots 1!}
\end{align*}
For example, if $n=3$, 
\begin{align*}
\text{\# of arr. where 2 people share a birthday}& =\text{\# of ways to select the unshared birthday}\\
& \quad\times \text{\# of ways to select the shared birthday}\\
& \quad\times\text{\# of ways to arrange 1 unique birthday and 2 same birthdays}\\
& =365\times 364\times \frac{3!}{2!1!}\\
& =398\,580
\end{align*}
and then
\begin{align*}
P(\{\text{at least 2 people share a birthday}\})&=\dfrac{|\{\text{2 people share a birthday}\}|+|\{\text{3 people share a birthday}\}|}{365^3}\\
& =\dfrac{398\,580+365}{365^3}\\
& =\dfrac{398\,945}{365^3}
\end{align*}
%\textsc{In general,}
%\begin{align*}
%|\{\text{at least 2\ldots}\}|& =(365\times 364\times\ldots\times(365-(n-2-1))\times(365-(n-2)))\times \dfrac{n!}{2!1!\ldots 1!}\\
%& \quad+(365\times 364\times\ldots\times(365-(n-3-1))\times(365-(n-3)))\times\dfrac{n!}{3!1!\ldots 1!}\\
%& \quad+\ldots+365\\
%& =\sum\limits_{i=2}^n \left(\dfrac{n!}{i!}\prod\limits_{j=i}^{n}(365-(n-j))\right)
%\end{align*}
%and so
%\begin{equation*}
%\boxed{
%P(\{\text{at least 2 people share a birthday}\})=\dfrac{\sum\limits_{i=2}^n \left(\dfrac{n!}{i!}\prod\limits_{j=i}^{n}(365-(n-j))\right)}{365^n}
%}
%\end{equation*}
\textsc{However}, this seems very laborious to compute, particularly if $n$ is large.\\
We can instead use the complement rule to determine that
\begin{equation*}
\boxed{\begin{split}
P(\{\text{at least 2 people share a birthday}\})& =1-P(\{\overline{\text{at least 2 people share a birthday}}\})\\
& =1-P(\{\text{no shared birthdays}\})\\
& =1-\dfrac{\text{\# of ways to select $n$ unshared birthdays}}{365^n}\\
& =1-\dfrac{365\times 364\times\ldots\times(365-n+1)}{365^n}
\end{split}}
\end{equation*}
For example, if $n=3$, 
\begin{align*}
P(\{\text{at least 2 people share a birthday}\})
& =1-\dfrac{\text{\# of ways to select 3 unshared birthdays}}{365^n}\\
& =1-\dfrac{365\times 364\times 363}{365^3}\\
& =1-\dfrac{48\,228\,180}{365^3}\\
& =\dfrac{398\,945}{365^3}
\end{align*}
\textsc{Notice} that, if $n>365$, the above calculation produces
\begin{align*}
P(\{\text{at least 2\ldots}\})& =1-\dfrac{365\times(365-1)\times\ldots\times(365-364)\times(365-365)\times\ldots\times(365-n+1)}{365^n}\\
& =1-\dfrac{365\times\ldots\times 0\times\ldots\times(365-n+1)}{365^n}\\
& =1-\dfrac{0}{365^n}\\
& =1-0=1
\end{align*}
Why does this make sense?\\[1ex]
By the pigeonhole principle, if we have $n$ objects to place in fewer than $n$ pigeonholes, at least 1 pigeonhole will contain multiple objects. In this case, if there are more than 365 birthdays to distribute over 365 days, at least 2 birthdays will fall on the same day. Thus, the probability of at least 2 people sharing a birthday is 1, or absolutely certain.\\[1ex]
\textsc{Note} that this counting method counts \textit{ordered} $n$-tuples: for example, where $n=3$, we consider (1, 1, 2) and (1, 2, 1) to be different combinations of birthdays.\\
If we were to instead consider unordered $n$-tuples, we could not use the classical definition of probability, since not all outcomes would be equally likely. For example, where $n=3$, the unordered combination (1, 1, 2) is more likely than the unordered combination (1, 2, 3), since there are more ways in which the former can occur.



\end{document}