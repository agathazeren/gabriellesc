\documentclass{article}

\usepackage{fullpage}
\usepackage{textcomp}
\usepackage{amsmath}

\usepackage{fancyhdr}
\pagestyle{fancy}
\renewcommand{\headrulewidth}{0pt}
\cfoot{\sc Page \thepage\ of \pageref{end}}

\begin{document}

{\large \noindent{}University of Toronto at Scarborough\\
\textbf{CSC A67/MAT A67 - Discrete Mathematics, Fall 2015}}

\section*{\huge Exercise \#8: Induction}

{\large Due: November 27, 2015 at 11:59 p.m.\\
This exercise is worth 3\% of your final grade.}\\[1em]
\textbf{Warning:} Your electronic submission on MarkUs affirms that this exercise is your own work and no
one else's, and is in accordance with the University of Toronto Code of Behaviour on Academic Matters,
the Code of Student Conduct, and the guidelines for avoiding plagiarism in CSC A67/MAT A67.\\[1ex]
This exercise is due by 11:59 p.m. November 27. Late exercises will not be accepted.\\[1ex]
\renewcommand{\labelenumi}{\arabic{enumi}.}
\renewcommand{\labelenumii}{(\alph{enumii})}
\begin{enumerate}
\item \begin{enumerate}
	\item Prove\marginpar{[6]}, without using induction, that $n(n+1)$ is an even number for every nonnegative integer $n$.
	\item Prove, using induction, that $n(n+1)$ is an even number for every nonnegative integer $n$.
	\end{enumerate}
\item Prove\marginpar{[3]} the following identity:
\begin{equation*}
1\cdot 2+2\cdot 3+3\cdot 4+\ldots+(n-1)\cdot n=\dfrac{(n-1)\cdot n\cdot(n+1)}{3}
\end{equation*}
\item Prove\marginpar{[3]} that the sum of the first $n$ squares ($1+4+9+\ldots+n^2$) is $\tfrac{n(n+1)(2n+1)}{6}$.
\item Prove\marginpar{[6]} by induction on $n$ that
	\begin{enumerate}
	\item $n^2-1$ is a multiple of 4 if $n$ is odd.
	\item $n^3-n$ is a multiple of 6 for every $n$.
	\end{enumerate}
\item Use\marginpar{[3]} induction on $n$ to prove that the number of handshakes between $n$ people is $\tfrac{n(n-1)}{2}$.
\item Read\marginpar{[2]} the following induction proof carefully:
	\begin{description}
	\item[Assertion:] $n(n+1)$ is an odd number for every $n$.
	\item[Proof:] Suppose that this is true for $n-1$. We have $n(n+1)=(n-1)n+2n$. Here $(n-1)n$ is odd by the induction hypothesis, and $2n$ is even. Hence $n(n+1)$ is the sum of an odd number and an even number, which is odd.
	\end{description}
The assertion that we proved is obviously wrong for $n=10$, since $10\cdot 11=110$ is even. Explain what is wrong with this proof.
\item Use\marginpar{[3]} induction to prove that $n!>2^n$ if $n\geq 4$.
\end{enumerate}
\hrulefill\\
\noindent[Total: 26 marks]\label{end}

\end{document}